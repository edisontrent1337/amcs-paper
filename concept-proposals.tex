\chapter{Redesign Strategy}
\label{chapter:redesignstrategy}
Several weaknesses and flaws of the web view of AMCS have been identified and analyzed in the previous chapter. They range from issues regarding visualization, layout and space usage to problems with user navigation.
A set of proposals that aim at solving these problems is introduced in the subsequent chapter. The focus will predominantly lie on using the available screen space more efficiently, improving local navigation inside polls and global navigation between different views and reducing indirection to a minimum. Each proposal for itself is centered around improving the usability of the application while all proposals as a whole aim at keeping the interface consistent and recognizable across all supported platforms.

\section{Main View}

Several improvements for the layout and visualization of the \emph{Main View} follow in this section. \Cref{figure:mainviewenhancement}, \Cref{figure:mainviewenhancement2} and \Cref{figure:mainviewenhancement3} respectively show an evolution of mock-ups for the \emph{Main View}.
\subsection{Layout}
\label{section:con:proposals:mainview:layout}
\paragraph{Iteration 1}
As mentioned in \Cref{section:con:problems:mainview:generalvis}, the \emph{Main View} suffers from using the available vertical space not effectively enough. Most noticeably, the course overview and enrollment form are placed below the list of lectures. In order to find information about relevant courses or to enroll / leave a course, students are required to scroll all the way to the bottom. 
Therefore, one proposal is to compress this view by using drop down menus and tabs. A mock-up of the proposals described in the following can bee seen in \Cref{figure:mainviewenhancement}. First of all, the view is restructured to follow the hierarchical concept of courses containing lectures: In the top (1), a button for a drop down menu is shown next to the currently selected courses' name and two additional buttons (6) and (7). The functionality of the drop down menu and the buttons is explained later.
Following the heading, tabs for \emph{past}, \emph{upcoming} and \emph{live} lectures are laid out side by side (2). In the first two mock-ups (see \Cref{figure:mainviewenhancement} and \Cref{figure:mainviewenhancement2} respectively), the tab bar is followed by the \emph{Navigation Bar}, a numbered horizontal list of clickable dots (3) that each represent one lecture. The currently selected lecture is highlighted with a bold blue border to enhance visibility and orientation. Finally, the information section of the view follows (4) with the title of the lecture, time and duration details and the lecture description. In the details section, an additional button is placed (5) that is labeled as “Evaluate”. Another idea would be to hide or deactivate this button for lectures that have not yet entered the state \emph{past}.
This layout uses a considerably reduced amount of vertical space. The placement of the “Evaluate” button (5) is motivated by further reducing vertically occupied space as much as possible, but it could be reasonable to place it below the description text of the lecture. On most devices, the amount of scrolling required should be reduced with the proposed layout. In an effort to keep visual clutter and redundancy at a minimum, the badges that indicate temporal context of the lectures now miss completely.

\paragraph{Iteration 2}
After collecting feedback from the AMCS group, the mock-up was adjusted slightly (see \Cref{figure:mainviewenhancement2}). The textual description of the tabs is replaced with iconography. Furthermore, \emph{Notification Bubbles} are introduced to indicate new or unread content. The \emph{Evaluate Button} was moved to a more conventional location at the bottom of the view. The number of simultaneously displayed lectures remains at one.

\paragraph{Iteration 3}
Regarding the strict reduction of vertical space used in the layout, one might suggest that the new \emph{Main View} has become too compact.
Feedback from the AMCS group led to the realization, that multiple lectures should be visible at the same time for the reason that typically multiple lectures are taking place on the same day.
Therefore, the next iteration expanded the layout again to display 3 lectures at once (see \Cref{figure:mainviewenhancement3}). A page view concept is proposed that minimizes scrolling and maximizes the amount of information displayed at once. The location of the \emph{Evaluate Button} has moved again to the top portion of a lecture box.

\begin{figure}[ht]
	\centering
	\begin{minipage}[t]{.7\textwidth}
		\includegraphics[width=\textwidth]{mockups/main_view_enhancement_v1_annotated.png}
			\captionsetup{width=.8\linewidth}
		\caption{Mock-up 1 of the new \emph{Main View}: Currently, the \emph{past courses} tab is shown.
		A course that has already taken place is selected.
		Usage of drop down menus and tabs to reduce the amount of vertical space used. (5), (6) and (7) serve as buttons to evaluate given answers for the shown lecture, go to the question pool of this course and unsubscribe from the selected course respectively.
		}
		\label{figure:mainviewenhancement}
	\end{minipage}%
\end{figure}
\begin{figure}[ht]
	\begin{minipage}[t]{\textwidth}
		\centering
		\includegraphics[width=.7\textwidth]{mockups/main_view_enhancement_v2_cropped.png}
				\captionsetup{width=.8\linewidth}
		\caption{Mock-up 2 of the new \emph{Main View}:
			Textual descriptions for tabs are replaced with iconography, notification
			bubbles for unread content are introduced. Colors are adjusted to match the corporate design of AMCS.
		}
		\label{figure:mainviewenhancement2}
	\end{minipage}
\end{figure}
\begin{figure}[ht]
	\begin{minipage}[t]{\textwidth}
		\centering
		\includegraphics[width=.7\textwidth]{mockups/main_view_enhancement_v3.png}
		\captionsetup{width=.8\linewidth}
		\caption{Mock-up 3 of the new \emph{Main View}:
			The icon indicating an ongoing lecture for a course is brought back.
			A combination of textual description and iconography is used for tabs.
			Multiple lectures are displayed at once in a page-based view.
		}
		\label{figure:mainviewenhancement3}
	\end{minipage}
\end{figure}

\subsection{Navigation}
\label{section:con:proposals:mainview:navigation}
Tabs (2) should separate lectures by their temporal context. Selecting a tab will only display lectures that share the respective temporal context, meaning that it should be easier to switch between \emph{past}, \emph{upcoming} and \emph{live} lectures.
The \emph{Navigation Bar} (3) in \Cref{figure:mainviewenhancement3} is used to ease navigation between lectures that share the same temporal context. A student can use the bar to switch quickly between the oldest and newest past lecture by selecting the corresponding button from the bar. This should improve navigation within the \emph{Main View} as well as in the context of a lecture. 
Additionally, this proposal aims at removing indirection in the global navigation context by providing certain buttons that serve as shortcuts for the functionality currently found in the \emph{Burger Menu}. Button (6) serves as a shortcut to the \emph{Question Pool} for the selected course. On click of button (7), the student will be removed from the selected course. Both of these buttons are placed in the header of the layout next to the course’s name to indicate that both referenced functionalities operate on a course scale, whereas the “Evaluate” button (5) operates on a lecture scale. The \emph{Evaluate Button} eliminates the need to select a course from a drop down menu and even choosing from a list of lectures is not necessary anymore, as all of these information needed to send the request to the backend are present. Therefore, multiple layers of indirection are removed from the established workflow. All three buttons try to remove the indirection introduced by the \emph{Burger Menu}ch as possible. Functionality that is associated with a course or lecture is triggered from a view that deals with courses or lectures respectively. The \emph{Burger Menu} would then only have to deal with the profile editing and logout functionalities.

\subsection{Embedded Drop Down Menu}
\label{section:con:proposals:dropdown}
The drop down menu is introduced to help reducing usage of vertical space even more (see \Cref{figure:embeddeddropdown}). Clicking on it reveals its two functions: Besides the now embedded enrollment form, a list of courses a student is already enrolled to is shown below. The enrollment still consists of a text field and a button. Both elements are shown next to the text \emph{Enroll...}. The close proximity to the list of enrolled courses could make this functionality potentially more traceable to users. The enrollment form will only be embedded in the drop down menu when the student is enrolled in at least one course beforehand. Otherwise, in place of the \emph{Main View}, only the enrollment form should be shown.
When selecting an item from list of courses below, corresponding lectures will be brought up in the \emph{Main View}.
In summary, the \emph{Drop Down Menu} acts as a filter to the \emph{Main View} and essentially covers the responsibilities of the \emph{Course View} currently in place.

\begin{figure}[ht]
	\begin{minipage}[t]{\textwidth}
		\centering
		\includegraphics[width=.7\textwidth]{mockups/embedded_drop_down.png}
		\captionsetup{width=.8\linewidth}
		\caption{Mock-up of the new \emph{Embedded Drop Down Menu}:
			The enrollment form is integrated in the menu. Users can change the currently selected course from here. The \emph{Main View} then adapts accordingly, allowing the menu to act as a filter for courses.
		}
		\label{figure:embeddeddropdown}
	\end{minipage}
\end{figure}

\subsection{Poll View}

\subsection{Layout and Visualization}
Several issues have been identified regarding layout and visualization of polls in \Cref{section:con:problems:pollview}. Main problems include the ineffective use of vertical space in this view and a lack of separation between types of polls. Regardless of their type, polls are simply appended to the bottom of the list, making it even longer.
In order to solve these issues, a tab-based layout is used once more. 
 \Cref{figure:pollviewenhanvement1} shows the first mock-up iteration for the redesign of the \emph{Poll View}.
\paragraph{Iteration 1}
Beginning at the top, the course's name is displayed in white font on a blue rectangle (1). Following up, to separate the type of polls from one another, a \emph{Tab Menu} is used (2) to differentiate between \emph{Slide Polls}, \emph{Lecture Polls} and \emph{Course Polls}. The tabs are arranged from left to right depending on the poll's lifetime. The most short lived polls, the \emph{Slide Polls} are placed on the left, the \emph{Lecture Polls} take advantage of the middle and the \emph{Course Polls} are displayed to the right. Selecting one of the tabs will cause the layout to show only polls of said type, making them act as a filter to what is currently displayed. This will potentially improve the effectiveness of vertical space used greatly. 

\begin{figure}[ht]
	\begin{minipage}[t]{\textwidth}
		\centering
		\includegraphics[width=.7\textwidth]{mockups/poll_view_enhancement.png}
		\captionsetup{width=.8\linewidth}
		\caption{Mock-up 1 of the new \emph{Poll View}:
			The icon indicating an ongoing lecture for a course is brought back.
			A combination of textual description and iconography is used for tabs.
			Multiple lectures are displayed at once in a page-based view.
		}
		\label{figure:pollviewenhanvement1}
	\end{minipage}
\end{figure}

\begin{figure}[ht]
	\begin{minipage}[t]{\textwidth}
		\centering
		\includegraphics[width=.7\textwidth]{mockups/poll_view_enhancement_v2.png}
		\captionsetup{width=.8\linewidth}
		\caption{Mock-up 2 of the new \emph{Poll View}:
			Colors are adapted to match the corporate design of AMCS.
		}
		\label{figure:pollviewenhanvement2}
	\end{minipage}
\end{figure}
Below the \emph{Tab Menu}, a \emph{Navigation Bar} (3) is displayed that contains the question's topic, index and the total number of questions. In case of a slide poll, the current slide number is shown additionally (see \Cref{figure:pollviewenhanvement1}).
Furthermore, the \emph{Navigation Bar} introduced in \Cref{section:con:proposals:mainview:navigation} is reused here (4). It serves as a means to navigate between questions more easily and faster but also reduces vertical space used significantly. Only one question at a time is displayed to the student. Appropriate colors and icons are intended to convey information more efficiently. A blue bold border is used to indicate the current question selected in the \emph{Navigation Bar}, light blue dots signify, that the corresponding question has not been answered yet, whereas bold green or red dots indicate correct and wrongly answered questions respectively. 
\newline
\newline
Below the \emph{Navigation Bar}, only one question at a time is displayed to the student to avoid visual noise and clutter (5). The question is displayed in a blue rectangle with white text. Below the question, an instance of an answering mechanism is displayed (6). 
Currently, each questions is answered individually by selecting the option and then pressing the blue \emph{Answer} button (see \Cref{figure:pollview}). Afterwards, feedback is shown immediately to the student. In terms of usabilty, users might find this tedious and redundant. One idea that comes to mind is to send the answers of a poll collectively in bulk to the server at the end of a poll. An advantage with this approach is the reduced amount of requests sent to the server. However, AMCS follows a rather strict principle of providing immediate feedback. Students should directly be informed about a the correctness of an answer.
Therefore, in the case of SC-, MC-, SCC- or MCC-question, the button to answer the question is omitted and merely selecting an option will trigger a request to the back end server.
Wrong answers are highlighted as before in red, a correct answer in green and it will still be possible to answer twice. Finally, space for textual feedback is given in a box (7) below the answers. This view is reusable and can therefore also be used to display already answered questions when using the \emph{Evaluate answers} functionality. This view will likewise profit from the reduced amount of vertical space used. 

\paragraph{Iteration 2}
Similar to the later iterations of the \emph{Main View} described earlier, the mock-up was adapted to comply with the corporate design of AMCS (see \Cref{figure:pollviewenhanvement2}).
The tabs at the top now use textual descriptions and iconography simultaneously to convey their meaning more efficiently, as feedback by the AMCS group led to the assumption that icons alone are not recognizable enough. The tabs are further enhanced by the inclusion of \emph{Notification Bubbles} that indicate the amount of unanswered question in each poll category, potentially reducing the cognitive effort to find unanswered questions.
The \emph{Navigation Bar} is further enhanced by using icons that represent the state of a specific question. Green arrows and red crosses are used to visualize correctly or wrongly answered questions respectively to improve accessibility, especially for colorblind students.
The buttons that allow to jump between questions are moved to the \emph{Navigation Bar}, as they resided previously next to the topic and question in order to separate navigation from content.
The indicator for the current slide number is modified slightly and pushed to the left. On the right side, icons that differentiate between question types are reintroduced. 
\subsection{Navigation between questions}

Navigation between questions should be made easier for students and focus on one question at a time. Therefore, an improvement would be to introduce two buttons in the \emph{Navigation Bar} that can be used to navigate one question forward or backwards. Pressing the respective button will cause to show the next or previous question, regardless of whether the current question has already been answered. This leads to the same level of freedom when navigating polls that the current state of the application allows.
\newline
\newline
Swiping is a widely spread way of interacting with a user interface on smartphones or tablets. Consequently, a student might expect to be able to use these gestures while using AMCS. Therefore, navigating between questions should be possible by swiping left to go forward or right to go backwards. The combination of swiping and the provision of buttons for navigation enhances usability while respecting different platforms and device types. Students on smartphones and tablets will have buttons and swipe gestures simultaneously available to them, while users on laptops and computers without touchscreens can use the buttons.
In addition to that, the student can use the indicators inside the \emph{Navigation Bar} to freely select a question they wish to answer or review. This eases navigation within a poll, no scrolling is required anymore.


\section{Course View}
The \emph{Course View} has an important function in terms of usability since it acts as a filter for lectures belonging to a certain course. Users must be given the opportunity to sort and filter a list of elements, which is why this function has to be preserved in the redesign. However, as described in \Cref{section:con:problems:courseview}, the \emph{Course View} has a redundant nature as it looks and feels nearly identical to the \emph{Main View}.
Furthermore, some potentially confusing click paths lead to the \emph{Course View} as elaborated in \Cref{section:con:problems:navigation}.
As outlined in \Cref{section:con:proposals:dropdown}, the \emph{Drop Down Menu} serves as a filter for courses, rendering the \emph{Course View} obsolete. It is therefore dropped by the redesign. This results in an additional side effect in form of reduced amounts of click paths and stronger interconnectedness between all the views (see \todosct).
\subsection{Navigation}
By shifting navigation elements from the \emph{Burger Menu} to components of the \emph{Main View}, the new \emph{Burger Menu} is slimmed down.
It will only contain the buttons labeled \emph{How it works}, \emph{Edit account} and \emph{Login/Logout}. The options to view the \emph{Question Pool}, to \emph{Evaluate} a given lecture or to unsubscribe from a course have all been moved to the \emph{Main View}, as described in  \Cref{section:con:proposals:mainview:layout} and \Cref{section:con:proposals:dropdown}.
The changes lead to shorter click paths in general with more well-defined behavior (see \todogrf).


\todo{Add a table that names and describes all proposals so that the implementation chapter can refer to this}