\section{Redesign Strategy}
\label{section:con:proposals}
Several weaknesses and flaws of the web view of AMCS have been identified and analyzed in the previous section. They range from issues regarding visualization, layout and space usage or user navigation.
A set of proposals that aims at solving these problems is introduced in the subsequent section. Mainly, the focus will lay on using the available screen space more efficiently, improving local navigation inside polls and global navigation between different views and reducing indirection to a minimum.

\subsection{Main View}

\subsubsection{Layout}
As mentioned in section \ref{section:con:problems:mainview:generalvis}, the \emph{Main View} suffers from using the vertical space available inefficiently. Most noticeably, the course overview and enrollment form are placed below the list of lectures. In order to find information about relevant courses or to enroll / leave a course, students are required to scroll all the way to the bottom. 
Therefore, one proposal is to compress this view by using drop down menus and tabs (see Fig \todosct). First of all, the view is restructured to follow the hierarchical concept of courses containing lectures: In the top (1), a button for a drop down menu is shown next to the currently selected courses' name and two additional buttons (6) and (7). The functionality of the drop down menu and the buttons is explained later.
Following the heading, selectable tabs for \emph{past}, \emph{upcoming} and \emph{live} lectures are laid out side by side (2). The tab bar is followed by a numbered horizontal list of clickable dots (3) that each represent one lecture. The currently selected lecture is highlighted with a bold blue border to enhance visibility and orientation. Finally, the information section of the view follows (4) with the title of the lecture, time and duration details and the lecture description. In the details section, an additional button is placed (5) that is labeled as “Evaluate”. This button is only available on past lectures, as shown in Figure \todosct.
\todogrf
This layout reduces the amount of vertical space used drastically. The placement of the “Evaluate” button (5) is motivated by reducing vertically occupied space as much as possible, but it could be reasonable to place it below the description text of the lecture. On most devices, the amount of scrolling required should be reduced with the proposed layout. As much redundancy as possible is eliminated from the view, as the badges that indicate temporal context of the lectures now miss completely.

\subsubsection{Navigation}
Tabs (2) should separate lectures by their temporal context. Selecting a tab will only display lectures that share the respective temporal context, meaning that it should be easier to switch between \emph{past}, \emph{upcoming} and \emph{live} lectures.
The \emph{Button Bar} (3) in Figure 5 is used to ease navigation between lectures that share the same temporal context. A student can use the bar for example to switch quickly between the oldest and newest past lecture by selecting the corresponding button from the bar. This should improve navigation within the \emph{Main View} as well as within a lecture context. 
Additionally, this proposal aims at removing indirection as much as possible on a global scale by providing certain buttons that serve as shortcuts for the functionality that is currently found in the \emph{Burger Menu}. Button (6) serves as a shortcut to the \emph{Question Pool} for the selected course. On click of button (7), the student will be removed from the selected course. Both of these buttons are placed in the header of the layout next to the course’s name to indicate that both referenced functionalities operate on a course scale, whereas the “Evaluate” button (5) operates on a lecture scale. The \emph{Evaluate Button} eliminates the need to select a course from a drop down menu and even choosing from a list of lectures is not necessary anymore, as all of these information needed to send the request to the backend are present. Therefore, multiple layers of indirection are removed from the established workflow. All three buttons try to remove the indirection introduced by the \emph{Burger Menu}ch as possible. Functionality that is associated with a course or lecture is triggered from a view that deals with courses or lectures respectively. The \emph{Burger Menu} would then only have to deal with the profile editing and  logout functionalities.

\subsubsection{Embedded Drop down menu}
\label{section:con:proposals:dropdown}
The drop down menu (see \todogrf) is introduced to help reducing usage of vertical space even more. Clicking on it reveals its two functions: for one, the enrollment form is now embedded in the drop down menu - a text field and a button are shown next to the text \emph{Enroll...}. The close proximity to courses the student already enrolled into makes this functionality potentially more intuitive to be found by users. The enrollment form will only be embedded in the drop down menu when the student is enrolled in at least one course beforehand. Otherwise, in place of the \emph{Main View}, only the enrollment form should be shown. Besides the enrollment form, a list of courses a student is already enrolled to is shown below. Selecting an item from this list will bring up the corresponding course view, similar to Figure \todosct \todogrf.
The \emph{Drop Down Menu} acts as a filter to the \emph{Main View} and essentially covers the responsibilties that the \emph{Course View} currently in place has.
\todo{Rewrite the sentence above. Its not true as we aim to eliminate the course view} An idea to further enhance the drop down menu would be to show notifications in little bubbles beside the course title to indicate new or upcoming lectures.


\subsection{Poll View}

\subsubsection{Layout}
Several issues have been identified regarding layout and visualization of polls in section \todosct. Main problems include the ineffective use of vertical space in this view and a lack of separation between types of polls. Different types of polls are simply appended to the bottom of the list, making it even longer. In order to solve these issues, some proposals that were made in \todosct can be applied here as well.  Figure \todosct shows an instance of a slide poll showing a SCC-question using a concept that is proposed in this section.
\todo{Rewrite the sentence above. Its bad}
Beginning at the top, the course's name is displayed in white font on a blue rectangle (1). Following up, to separate the type of polls from one another, a tab based menu is used (2) that differentiates between slide polls, lecture polls and course polls. Each tab therefore corresponds to a poll type. Selecting one of the tabs will cause the layout to show only questions of said type. This will improve the effectiveness of vertical space used greatly. The tabs act as a filter to what is currently displayed.

A further enhancement would be to show notifications in little bubbles besides the name of a tab to indicate new content.
\todo{Find a reference in real life to justify this statement}
 Below the tab menu, a navigation bar (3) is displayed that contains the name of the current topic and additional information such as the number of the current question. In case of a slide poll as shown in Figure 7, the current slide number is shown as well.
Also, the dot bar of section 3.2.1 is reused here (4). The dot bar serves as a means to navigate between questions more easily and faster but also reduces vertical space used significantly. It uses color coding and icons to convey information to the student. A blue bold border is used to indicate the current question selected in the dot bar, light blue dots signify, that the corresponding question has not been answered yet, where as bold green  or red dots indicate correct and wrongly answered questions respectively. The dot bar can be further enhanced by using icons that represent the state of a specific question. Green arrows and red crosses can be used to visualize correctly or wrongly answered questions respectively to ease usability for colorblind students. Below the dot bar, only one question at a time is displayed to the student to avoid overwhelming them (5). The question is displayed in a blue rectangle with white text. Below the question, an instance of an answering mechanism is displayed (6). In the case of SC-, MC-, SCC- or MCC-question, the button to answer the question is omitted. Multiple ideas for triggering the request to the AMCS backend exist here: Either selecting the answer will trigger a modal that asks the user whether or not he is sure with his choice of answer, and confirming this dialog will send the data to the server, or the mere selection of an answer will trigger the request.  A wrong answer is highlighted as before in red, a correct answer in green and it will still be possible to answer twice. Finally, space for textual feedback is given in a box (7) below the answers. This view is reusable and can therefore can also be used to display already answered questions when using the “Evaluate answers” functionality. This view then will also profit from the reduced amount of vertical space used. 

\subsubsection{Navigation between questions}

Navigation between questions should be made easier for students and focus on one question at a time. Therefore, an improvement would be to introduce two buttons in the navigation bar that can be used to navigate one question forward or backwards. Pressing the respective button will cause to show the next or previous question, regardless of whether the current question has already been answered. This leads to the same level of freedom when navigating polls that the current state of the application allows.

Swiping is a widely spread way of interacting with a user interface on smartphones or tablets. Consequently, a student might find it intuitive to use these gestures while using AMCS. Navigating between questions should be possible by swiping left to go forward or right to go backwards. The combination of swiping and the provision of buttons to navigate in the navigation bar helps not to break uniformity between different platforms. This will result in the fact that students on smartphones and tablets have buttons and swipe gestures available to them, while users on laptops and PCs without touchscreens can use the buttons.
In addition to that, the student can use the dot bar to freely select a question they wish to answer or review. This eases navigation within a poll, no scrolling is required anymore.


\subsection{Course View}
\todo{Write about how the course view is eliminated}
The \emph{Course View} has an important function in terms of usability since it acts as a filter for lectures belonging to a certain course. Users must be given the opportunity to sort and filter a list of elements, which is why this function has to be preserved in the redesign. However, as described in \ref{section:con:problems:courseview}, the \emph{Course View} has a redundant nature as it looks and feels nearly identical to the \emph{Main View}.
Furthermore, some potentially confusing click paths lead to the \emph{Course View} as elaborated in section \ref{section:con:problems:navigation}.
As outlined in segment \ref{section:con:proposals:dropdown}, the \emph{Drop Down Menu} serves as a filter for courses, rendering the \emph{Course View} obsolete. It is therefore dropped by the redesign. This results in an additional side effect in form of reduced amounts of click paths and stronger interconnectedness between all the views (see Figure \todosct).
\subsection{Navigation}
\todo{Write about how the naviagtion and the burger menu was slimmed}
\todo{Add a graphic of the new click paths}

\todo{Add a table that names and describes all proposals so that the implementation chapter can refer to this}