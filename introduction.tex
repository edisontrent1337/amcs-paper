
\chapter{Introduction}

\section{Motivation}


AMCS (Auditorium Mobile Classroom Service) is an Audience Response System (ARS) currently under development at the TU Dresden. Several members of the Faculty of Computer Science initialized the project in 2012 and since then, the system and its community have grown continuously and countless features were added over the years.
\newline
\newline
In general, the system's main objective is to improve the way knowledge is transferred from a speaker to their audience. By providing interactive polls and evaluation mechanisms, AMCS hopes to increase the audience's engagement before, during and after a lecture takes place, overall strengthening the interaction between speaker and audience.
\newline
\newline
On the one hand, several features of AMCS are designed to support speakers in their role. Regarding the educational context, docents at university use it not only during their lecture to get immediate feedback from participating students. Furthermore, it is used during the whole semester to prepare and manage lectures, to analyze learning progress and to evaluate feedback given by students. More specifically, a speaker can gain insight on the understanding of their audience by evaluating answers to their polls. For example, poor results to a poll covering a certain topic might suggest that the topic was misunderstood or badly explained. AMCS enables the speaker to use precisely this information, helping to identify parts of their lectures that students might struggle with. Speakers can focus more easily on repeating and emphasizing these topics in the future.
\newline
\newline
On the other hand, AMCS provides a set of features to the audience that aim at increasing their engagement.
The system aims at supporting a student's learning process by providing interactive polls, question pools and self-evaluation mechanisms that work on saved answers. While the polls are often used during lectures, the latter two tools can be used afterwards to identify and repeat difficult questions, to prepare for the next lesson or to study for the examination. 
\newline
\newline
Several standalone front end applications for different platforms such as iOS, Android and web are provided to the audience. However AMCS is used by the majority of students via its web page across different mobile devices such as laptops, tablets and smartphones, likely because it is very easy to access.
\newline
\newline 
Regarding the web page being the most common way AMCS is used, 
the challenge lays in providing a unified and responsive web-based user interface across all aforementioned device types. The motivation of this work is to analyze the application's state in terms of usability, design and consistency, identify weaknesses in each category and provide a set of proposals that hope to improve the design currently in place.

\section{Objective}

The main objective of this work is to provide a redesign strategy that - when implemented - improves the web view of AMCS on mobile devices in order to create a more user friendly experience.
Each proposal for itself is focused at improving usability of the application while all proposals as a whole aim at keeping a consistent and recognizable interface across all supported platforms. 
\newline
\newline
In order to reach this objective, an analysis of the current state of the application is conducted in chapter \ref{chapter:stateoftheart}, identifying weaknesses and inconsistencies that the design can be improved on.
\newline
\newline
Additionally, section \todosct covers relevant existing applications and work are analyzed to identify strategies that can be transferred and applied to AMCS. 
\newline
\newline
Finally, in section \todosct the strategy it is implemented as a prototype that uses the existing back end system. In section \todosct An evaluation of the prototype concludes this work.