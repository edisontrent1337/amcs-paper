%-----------------------------------------------------------------------------
%
%               Template for sigplanconf LaTeX Class
%
% Name:         sigplanconf-template.tex
%
% Purpose:      A template for sigplanconf.cls, which is a LaTeX 2e class
%               file for SIGPLAN conference proceedings.
%
% Guide:        Refer to "Author's Guide to the ACM SIGPLAN Class,"
%               sigplanconf-guide.pdf
%
% Author:       Paul C. Anagnostopoulos
%               Windfall Software
%               978 371-2316
%               paul@windfall.com
%
% Created:      15 February 2005
%
%-----------------------------------------------------------------------------


\documentclass{sigplanconf}

% The following \documentclass options may be useful:

% preprint      Remove this option only once the paper is in final form.
% 10pt          To set in 10-point type instead of 9-point.
% 11pt          To set in 11-point type instead of 9-point.
% authoryear    To obtain author/year citation style instead of numeric.

\usepackage{amsmath}


\begin{document}

\special{papersize=8.5in,11in}
\setlength{\pdfpageheight}{\paperheight}
\setlength{\pdfpagewidth}{\paperwidth}

\conferenceinfo{CONF 'yy}{Month d--d, 20yy, City, ST, Country} 
\copyrightyear{20yy} 
\copyrightdata{978-1-nnnn-nnnn-n/yy/mm} 
\doi{nnnnnnn.nnnnnnn}

% Uncomment one of the following two, if you are not going for the 
% traditional copyright transfer agreement.

%\exclusivelicense                % ACM gets exclusive license to publish, 
                                  % you retain copyright

%\permissiontopublish             % ACM gets nonexclusive license to publish
                                  % (paid open-access papers, 
                                  % short abstracts)

\titlebanner{banner above paper title}        % These are ignored unless
\preprintfooter{short description of paper}   % 'preprint' option specified.

\title{Redesign and unification of AMCS and its mobile web view}

\authorinfo{Sinthujan Thanabalasingam}
           {TU Dresden}
           {sinthujan.thanabalasingam@tu-dresden.de}

\maketitle

\begin{abstract}

AMCS is an Audience Response system developed by several individuals at the TU Dresden in 2012. Speakers and docents can use it to get immediate feedback from students that participate in live lectures. Its goal is to optimize the way knowledge is presented and transferred to the audience by offering polls for the audience to participate in before, during and after a lecture.
For the audience, the system provides several standalone frontend applications for different platforms such as iOS or Android and web. But because of its ease of access, the system is used by the majority of students via its web component across different mobile devices such as laptops, tablets and smartphones. 
Regarding usability, design and consistency, the challenge lays in these components to provide a unified user interface across all platforms that is intuitive to use.
\end{abstract}

TODO: fix the category and terms
\category{UI-design}{usability}{web}{}

% general terms are not compulsory anymore, 
% you may leave them out
\terms
term1, term2

\keywords
keyword1, keyword2

\section{Introduction}

\subsection{Motivation}


AMCS is an Audience Response system developed by several individuals at the TU Dresden in 2012. Speakers and docents can use it to get immediate feedback from students that participate in live lectures. Its goal is to optimize the way knowledge is presented and transferred to the audience by offering polls for the audience to participate in before, during and after a lecture.
For the audience, the system provides several standalone frontend applications for different platforms such as iOS or Android and web. But because of its ease of access, the system is used by the majority of students via its web component across different mobile devices such as laptops, tablets and smartphones. 
Regarding usability, design and consistency, the challenge lays in these components to provide a unified user interface across all platforms that is intuitive to use.

\subsection{Objectives}

The main objective of this work is to develop a redesign strategy that - when implemented - improves the highly used web view of AMCS that is currently in place.
Each proposal for itself is focused at improving usability of the application while all proposals as a whole aim at keeping a consistent and recognizable interface across all supported platforms. 
In order to reach this objective, an analysis of the current state of the application is conducted, identifying weaknesses and inconsistencies that the design can be improved on.
Additionally, relevant existing applications and work are analyzed in order to 

develop a prototype that uses the existing back end system.


\appendix
\section{Appendix Title}

This is the text of the appendix, if you need one.

\acks

Acknowledgments, if needed.

% We recommend abbrvnat bibliography style.

\bibliographystyle{abbrvnat}

% The bibliography should be embedded for final submission.

\begin{thebibliography}{}
\softraggedright

\bibitem[Smith et~al.(2009)Smith, Jones]{smith02}
P. Q. Smith, and X. Y. Jones. ...reference text...

\end{thebibliography}


\end{document}

%                       Revision History
%                       -------- -------
%  Date         Person  Ver.    Change
%  ----         ------  ----    ------

%  2013.06.29   TU      0.1--4  comments on permission/copyright notices

