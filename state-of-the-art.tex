\chapter{State of the art}

At the time of writing, AMCS offers front end applications for iOS, Android and web that students can use. As already mentioned in \todo INSERT HERE this work will focus on the web view / web page of AMCS.
This section elaborates the current state of the system by identifying and analyzing all views that allow for access to the different functionalities of AMCS. 


\section{Landing Page}
When accessing the website https://amcs.website, users will be shown the landing page of AMCS.
In order to use the system, users have to create an account by providing credentials. Additionally, users have to subscribe to lectures by typing in an optional PIN code. From thereon, students gain access to the system. 
\todogrf
 
\section{Main View}
After successfully logging in, the user is presented with the main view of the system.
It can be scrolled in the vertical direction and is divided into header and body. On top, the header comes along with corporate branding on the left side and a burger menu on the right side. Below it, the view's body organizes information in different sections as follows:

\todogrf

\subsection{Lectures}

This section lists all lectures that the user subscribed to. It is organized in subsections that indicate the temporal context of each lecture. These include:

\paragraph{Upcoming lectures} - Lectures that will take place in the future are shown here.
\paragraph{Active lectures} - Lectures that take place right now are shown here.
\paragraph{Past lectures} - Lectures that have already taken place are shown here.

\paragraph{Rendering of lectures}

Each of these subsections is organized as a vertical list that contains all corresponding lectures (see Fig.1). A lecture is rendered in a box that uses all horizontal space available to it. The box consists of a blue header with the lecture's name, a white info/detail area and a light blue footer that contains the course this lecture belongs to.
A color-coded badge on the top right of the boxes additionally serves as an indicator for the temporal context of the lecture.

\subsection{Enrolled Courses}

This section serves two purposes: Primarily, it provides a way to enroll into a course. An enrollment form is shown that consists of a text field to enter the course PIN and a blue button to trigger the enrollment (see Fig.1). When provided with a valid PIN, pressing the button redirects the user to a course overview on successful enrollment.
Secondly, the view shows all courses the student is currently enrolled in. They are rendered as light blue buttons in a vertical list. A trash can icon on each button provides a way to leave the given course.


\section{Poll View}
Answering polls is the main functionality of the system that users will engage with.
Polls can be reached by clicking on a lecture box as described in \todosct.
Each poll consists of a set of questions the user can give an answer to. They are rendered in a view that is reused  by the system depending on the situation and context. This means that the view might only be accessible under certain circumstances, for example when the lecture reaches a specific point in time, making it a slide poll (SP). SPs are shown when a specific slide is on display and can only be answered in this very moment. Other types of polls include “global” course polls (CP) that are always accessible during the semester and lecture polls (LP) that can only be answered during the life time of a lecture. The different types of polls that occur in AMCS are further summarized in Table \ref{tab:pollTypes}.

\begin{table}[t]
	{\renewcommand{\arraystretch}{2}L
		\begin{tabular}{ | p{3cm} | p{12cm} |}
			\hline
			Poll Type & Explanation \\ \hline \hline
			Slide Poll (SP) & Active when a specific slide is shown. Commonly used for pop quizzes after a difficult section in a lecture to make sure that students understood everything correctly. \\ \hline
			Preparation Poll (PP) & Active before the lecture takes place. Is commonly used to instruct students to prepare for a certain topic \\ \hline
			Lecture Poll (LP) & Active during the life time of a lecture. \\ \hline
			Post Processing Poll (PPP) & Active after a lecture has taken place. Commonly used to check gained knowledge. \\ \hline
			Course Poll (CP) & Active during the whole lifetime of the course (commonly during the whole semester) \\
			\hline
		\end{tabular}
	}
	\caption{Different poll types that the user might encounter when using AMCS.}
	\label{tab:pollTypes}
\end{table}

\section{Menu and Navigation}
Besides by using the main view, additional functionality can be reached by navigating the burger menu that is shown on the top right of the screen. It reveals a sub menu that expands vertically on the view, offering three additional sub menus. In the following, these sub menus and their functionality are briefly explained.


\subsection{Student}

This is one of the most important buttons that offers access to a subset of main functionalities of AMCS. Upon pressing this button, the menu expands again vertically, showing a list of further sub menus. Most of the functionalities shown in this list will be touched by the proposals for improvement that are presented section \todosct. The functionalities in questions are:

\begin{enumerate}
	\item Question Pool
	\item Evaluation of answers
	\item Edit account
\end{enumerate}


\subsection{How it works}

Pressing this button will redirect to a page that shows tutorial instructions on how to use AMCS.
This help page is rendered identical on all mobile devices and therefore falls out of the scope of this paper.

\subsection{Logout}

As the name already states, pressing this button will logout the user and end the session. 
If the logout was successful, the landing page of AMCS is shown.



\subsection{Question Pool}

By selecting this option from the burger menu, the student is offered the possibility to create collections of already answered questions. The intent is to provide a way for the students to identify and repeat questions that they had difficulty in answering.
Similar to 2.1.2, the student is prompted with a drop down menu to select the course they are interested in. After selection, the student is presented with a list of all lectures and their polls respectively. All questions of each poll are grouped and shown to the student in a vertical list. From this list, the student can select all questions that they might be interested in to create a pool of questions.
These pools are composed into polls that the student then can answer again. These polls are rendered in the same manner as stated in 2.3.


\section{Related Systems}
\todo{Add this}